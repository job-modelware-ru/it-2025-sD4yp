\documentclass[12pt]{article}
\usepackage[utf8]{inputenc}
\usepackage[russian]{babel}
\usepackage[T2A]{fontenc}
\usepackage{geometry}
\usepackage{array}
\usepackage{multirow}
\usepackage{booktabs}
\usepackage{longtable}
\usepackage{hyperref}
\usepackage{titlesec}
\usepackage{setspace}

% Настройки полей
\geometry{a4paper, margin=2cm}

% Межстрочный интервал
\onehalfspacing

\begin{document}

\section{Сценарии использования}

\subsection*{Акторы}
\item \textbf{Игрок} — конечный пользователь, ищущий игры.
    \item \textbf{Разработчик} — автор или издатель игр.
    \item \textbf{Менеджер платформы} — модератор и администратор сервиса.
    \item \textbf{Система} — рекомендательный движок и платформа.

\subsection{Сценарий 1: Получение персонализированных рекомендаций}
\begin{longtable}{|>{\raggedright\arraybackslash}p{2.5cm}|>{\raggedright\arraybackslash}p{3cm}|>{\raggedright\arraybackslash}p{9cm}|}
    \hline
    \textbf{Шаг} & \textbf{Актор} & \textbf{Действие} \\
    \hline
    1 & Игрок & Регистрируется на платформе и вводит игровые предпочтения (жанры, платформы, механики). \\
    \hline
    2 & Система & Сохраняет профиль пользователя и активирует рекомендательный алгоритм. \\
    \hline
    3 & Система & Анализирует поведение пользователя (просмотры, покупки, время в игре). \\
    \hline
    4 & Система & Формирует список персонализированных рекомендаций. \\
    \hline
    5 & Игрок & Просматривает рекомендации на главной странице или в личном кабинете. \\
    \hline
    6 & Игрок & Взаимодействует с рекомендациями: открывает карточки, читает описания. \\
    \hline
    7 & Игрок & Оценивает рекомендации (лайк/дизлайк). \\
    \hline
    8 & Система & Обновляет профиль пользователя и корректирует будущие рекомендации. \\
    \hline
    \multicolumn{3}{|l|}{\textit{Постусловие: Профиль пользователя становится точнее, качество рекомендаций растёт.}} \\
    \hline
\end{longtable}

\subsection{Сценарий 2: Добавление игры разработчиком}
\begin{longtable}{|>{\raggedright\arraybackslash}p{2.5cm}|>{\raggedright\arraybackslash}p{3cm}|>{\raggedright\arraybackslash}p{9cm}|}
    \hline
    \textbf{Шаг} & \textbf{Актор} & \textbf{Действие} \\
    \hline
    1 & Разработчик & Авторизуется в личном кабинете. \\
    \hline
    2 & Разработчик & Заполняет форму: название, жанр, описание, скриншоты, цена, ссылка. \\
    \hline
    3 & Система & Проверяет данные и сохраняет черновик. \\
    \hline
    4 & Менеджер платформы & Проводит модерацию контента. \\
    \hline
    5 & Система & После одобрения публикует игру и включает её в систему рекомендаций. \\
    \hline
    6 & Система & Начинает сбор статистики по взаимодействию игроков с игрой. \\
    \hline
    \multicolumn{3}{|l|}{\textit{Постусловие: Игра доступна для рекомендаций; начинается сбор аналитики.}} \\
    \hline
\end{longtable}

\subsection{Сценарий 3: Получение аналитики разработчиком}
\begin{longtable}{|>{\raggedright\arraybackslash}p{2.5cm}|>{\raggedright\arraybackslash}p{3cm}|>{\raggedright\arraybackslash}p{9cm}|}
    \hline
    \textbf{Шаг} & \textbf{Актор} & \textbf{Действие} \\
    \hline
    1 & Разработчик & Заходит в личный кабинет. \\
    \hline
    2 & Разработчик & Переходит в раздел «Аналитика». \\
    \hline
    3 & Система & Отображает отчёты: показы, клики, добавления, покупки, средняя оценка. \\
    \hline
    4 & Система & Предоставляет данные о демографии аудитории и источниках переходов. \\
    \hline
    5 & Разработчик & Экспортирует отчёт в CSV или PDF. \\
    \hline
    \multicolumn{3}{|l|}{\textit{Постусловие: Разработчик получает данные для оптимизации маркетинга.}} \\
    \hline
\end{longtable}

\subsection{Сценарий 4: Модерация контента менеджером}
\begin{longtable}{|>{\raggedright\arraybackslash}p{2.5cm}|>{\raggedright\arraybackslash}p{3cm}|>{\raggedright\arraybackslash}p{9cm}|}
    \hline
    \textbf{Шаг} & \textbf{Актор} & \textbf{Действие} \\
    \hline
    1 & Система & Автоматически помечает подозрительные материалы. \\
    \hline
    2 & Менеджер платформы & Получает уведомление о новых заявках или жалобах. \\
    \hline
    3 & Менеджер платформы & Проверяет контент на соответствие политике. \\
    \hline
    4 & Менеджер платформы & Принимает решение: одобрить / отклонить / запросить правки. \\
    \hline
    5 & Система & Обновляет статус игры и уведомляет разработчика. \\
    \hline
    \multicolumn{3}{|l|}{\textit{Постусловие: Контент соответствует стандартам качества и безопасности.}} \\
    \hline
\end{longtable}
\newpage
\subsection{Сценарий 5: Обратная связь от игрока}
\begin{longtable}{|>{\raggedright\arraybackslash}p{2.5cm}|>{\raggedright\arraybackslash}p{3cm}|>{\raggedright\arraybackslash}p{9cm}|}
    \hline
    \textbf{Шаг} & \textbf{Актор} & \textbf{Действие} \\
    \hline
    1 & Игрок & Просматривает рекомендованную игру. \\
    \hline
    2 & Игрок & Нажимает «Нравится» или «Не нравится». \\
    \hline
    3 & Система & Фиксирует реакцию и обновляет весовые коэффициенты в алгоритме. \\
    \hline
    4 & Игрок & Оставляет отзыв и оценку после игры. \\
    \hline
    5 & Система & Интегрирует отзыв в статистику и учитывает при формировании рекомендаций. \\
    \hline
    \multicolumn{3}{|l|}{\textit{Постусловие: Улучшается точность рекомендаций для всех пользователей.}} \\
    \hline
\end{longtable}

\end{document}